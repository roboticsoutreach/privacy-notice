\documentclass[12pt]{article}
\usepackage[utf8]{inputenc}
\usepackage[hidelinks]{hyperref}
\usepackage{fullpage}
\usepackage{color}

%Remove all section numbering
\setcounter{secnumdepth}{0}

% Sans-Serif Font
\renewcommand\familydefault{\sfdefault}

\begin{document}

\begin{titlepage}
\begin{center}
  \bfseries
  \huge Southampton University Cyber Security Society
  \vskip.1in
  \vskip 3in
  \huge Privacy Notice
  \vskip 1in
\vskip 2.6in

\end{center}

\begin{center}
\large Revision 1.0 -  July 2018
\end{center}

\end{titlepage}

\newpage

\tableofcontents

\newpage


\section{What is the purpose of this document?}
The Southampton University Cyber Security Society is committed to protecting the privacy and security of your personal information.
This privacy notice describes how we collect and use personal information about you during and after your working relationship with us, in accordance with the General Data Protection Regulation (GDPR).
It applies to all members of the organisation.

The Southampton University Cyber Security Society is a "data controller". This means that we are responsible for deciding how we hold and use personal information about you. We are required under data protection legislation to notify you of the information contained in this privacy notice.
The person responsible for data protection within our organisation is the President who can be contacted at \href{mailto:sucss@soton.ac.uk?subject=Privacy Notice Enquiry}{sucss@soton.ac.uk}.

This notice applies to current and former members. This notice does not form part of any contract to provide services. We may update this notice at any time.  When changes are made to this document we will inform you electronically of the update.
It is important that you read this notice, together with any other privacy notice we may provide on specific occasions when we are collecting or processing personal information about you, so that you are aware of how and why we are using such information.
\vskip.1in

\section{Data protection principles}
We will comply with data protection law. This says that the personal information we hold about you must be:

\begin{enumerate}
\item Used lawfully, fairly and in a transparent way.
\item Collected only for valid purposes that we have clearly explained to you and not used in any way that is incompatible with those purposes.
\item Relevant to the purposes we have told you about and limited only to those purposes.
\item Accurate and kept up to date.
\item Kept only as long as necessary for the purposes we have told you about.
\item Kept securely.
\end{enumerate}

\subsection{The kind of information we hold about you}
Personal data, or personal information, means any information about an individual from which that person can be identified. It does not include data where the identity has been removed (anonymous data).
There are "special categories" of more sensitive personal data which require a higher level of protection.
We may collect, store, and use the following categories of personal information about you:

\begin{itemize}
\item Personal contact details such as name, title, addresses, telephone numbers, and personal email addresses.
\item Student ID number.
\item Date of birth.
\item Gender.
\item Next of kin and emergency contact information.
\item Information about your use of our information and communications systems.
\item Photographs.
\item Passwords (in some cases, if you have registered for an account on our web services).
We may also collect, store and use the following "special categories" of more sensitive personal information:
\item Information about criminal convictions and offences.

\end{itemize}

\section{Collection of personal information}

\subsection{How is your personal information collected?}

We typically collect personal information about members through the membership joining process directly from members. This is done both through our relationship with University of Southampton Students' Union, and through signups to our mailing list.We may sometimes collect additional information from third parties including the University of Southampton and the University of Southampton Students’ Union. 
We may collect additional personal information in the course of member-related activities throughout the period of your membership with us. Particular examples of this are signing up to events, or registering for an account on one of our web services.

\section{Use of personal information}


\subsection{How we will use information about you}
We will only use your personal information when the law allows us to. Most commonly, we will use your personal information in the following circumstances:

\begin{enumerate}
\item Where we need to perform the contract we have entered into with you.
\item Where we need to comply with a legal obligation.
\item Where it is necessary for our legitimate interests (or those of a third party) and your interests and fundamental rights do not override those interests. 
\item Where you are a significant member of the organisation that we need to introduce to a customer or other third party.
\end{enumerate}

We may also use your personal information in the following situations, which are likely to be rare:

\begin{enumerate}
\item Where we need to protect your interests (or someone else's interests).
\item Where it is needed in the public interest or for official purposes.
\end{enumerate}

\subsection{Situations in which we will use your personal information}

We need all the categories of information in the list above (see “The kind of information we hold about you”) primarily to allow us to perform our contract with you and to enable us to comply with legal obligations. In some cases we may use your personal information to pursue legitimate interests of our own or those of third parties, provided your interests and fundamental rights do not override those interests. The situations in which we will process your personal information are listed below. 

\begin{itemize}
\item Administering the contract we have entered into with you.
\item Maintaining your personal details (e.g. your name, photograph, membership number and preferred contact details), including ensuring effective communications with you.
\item Keeping financial records.
\item Assessing qualifications for a particular task.
\item Maintaining a formal record of your activities with us.
\item Training and development requirements.
\item Dealing with other third parties to whom your identity and background information is important. 
\item Dealing with legal disputes involving you, or other members, including accidents.
\item Complying with health and safety obligations.
\item To prevent fraud.
\item To ensure network and information security, including preventing unauthorised access to our computer and electronic communications systems and preventing malicious software distribution.
\item Equal opportunities monitoring.
\item Managing society alumni relations and fundraising.
\item Managing complaints made to us.
\item Some of the above grounds for processing will overlap and there may be several grounds which justify our use of your personal information. 
\end{itemize}

\subsection{If you fail to provide personal information}

If you fail to provide certain information when requested, we may not be able to perform the contract we have entered into with you (such as providing access to our activities), or we may be prevented from complying with our legal obligations (such as to ensure the health and safety of our members).


\subsection{Change of purpose}

We will only use your personal information for the purposes for which we collected it, unless we reasonably consider that we need to use it for another reason and that reason is compatible with the original purpose. If we need to use your personal information for an unrelated purpose, we will notify you and we will explain the legal basis which allows us to do so. 
Please note that we may process your personal information without your knowledge or consent, in compliance with the above rules, where this is required or permitted by law.

\section{Sensitive Personal Information}

\subsection{How we use particularly sensitive personal information}

"Special categories" of particularly sensitive personal information require higher levels of protection. We need to have further justification for collecting, storing and using this type of personal information. 

We may process special categories of personal information in the following circumstances:

\begin{enumerate}
\item In limited circumstances, with your explicit written consent.
\item Where we need to carry out our legal obligations and in line with our privacy standard.
\item Where it is needed in the public interest, such as for equal opportunities monitoring or in relation to our occupational pension scheme, and in line with our privacy standard.
\item Where it is needed to assess your working capacity on health grounds, subject to appropriate confidentiality safeguards.
Less commonly, we may process this type of information where it is needed in relation to legal claims or where it is needed to protect your interests (or someone else's interests) and you are not capable of giving your consent, or where you have already made the information public. We may also process such information about members or former members in the course of legitimate business activities with the appropriate safeguards.
\end{enumerate}

\subsection{Do we need your consent?}
We do not need your consent if we use special categories of your personal information in accordance with our written policy to carry out our legal obligations or exercise specific rights in the field of law. In limited circumstances, we may approach you for your written consent to allow us to process certain particularly sensitive data. If we do so, we will provide you with full details of the information that we would like and the reason we need it, so that you can carefully consider whether you wish to consent. 

\subsection{Information about criminal convictions}
We may only use information relating to criminal convictions where the law allows us to do so. This will usually be where such processing is necessary to carry out our obligations and provided we do so in line with our privacy standard.
Less commonly, we may use information relating to criminal convictions where it is necessary in relation to legal claims, where it is necessary to protect your interests (or someone else's interests) and you are not capable of giving your consent, or where you have already made the information public. 
We may also process such information about members or former members in the course of legitimate membership activities with the appropriate safeguards.

We envisage that we [will/will not] hold information about criminal convictions. 
We will only collect information about criminal convictions if it is appropriate given the nature of the role and where we are legally able to do so. Where appropriate, we will collect information about criminal convictions where we need that information because of your role or our activities.  We may be notified of such information directly by you in the course of you membership with us. 
We are allowed to use your personal information in this way to carry out our obligations.
Data sharing
We may have to share your data with third parties, including customers, third-party service providers and other entities in the group.
We require third parties to respect the security of your data and to treat it in accordance with the law.
We may transfer your personal information outside the EU for any of the purposes described in this notice.
If we do, you can expect a similar degree of protection in respect of your personal information.

\section{Access by third parties}

\subsection{Why might you share my personal information with third parties?}

We may share your personal information with third parties where required by law, where it is necessary to administer the relationship with you or where we have another legitimate interest in doing so. \textbf{We may be subject to a legal requirement (with or without your consent) to share your personal information with the University of Southampton, University of Southampton Students’ Union or a government agency} (such as the police or security services or other statutory authorities with investigatory powers) under special circumstances (e.g. relating to tax, crime or health and safety).  Where feasible and appropriate, we will notify you of our intention to share such information in advance.  

\subsection{Which third-party service providers process my personal information?}

"Third parties" includes third-party service providers (including contractors and designated agents) and other entities within our group. 
\begin{itemize}
\item We will share personal information with the \textbf{University of Southampton Students’ Union} as part of being an affiliated club or society, in order to process your membership, and to run the club or society.
\item We share some of your personal information with the \textbf{University of Southampton}, for access to physical and networked resources.
\item We share some of your personal information with the \textbf{University of Southampton, School of Electronics and Computer Science}, through which we run our mailing list. Further details of your membership to this list can be found at \url{http://mailman.ecs.soton.ac.uk/mailman/listinfo/sucss}.
\end{itemize}

\subsection{How secure is my information with third-party service providers}

All our third-party service providers and other entities in the group are required to take appropriate security measures to protect your personal information under the general law or in line with our policies. We do not allow our third-party service providers to use your personal data for their own purposes. We only permit them to process your personal data for specified purposes and in accordance with our instructions.

\section{Data retention}
We will only retain your personal information for as long as necessary to fulfil the purposes we collected it for, and usually for [number of years] after your membership ceases. This may include for the purposes of satisfying any legal, accounting, or reporting requirements. To determine the appropriate retention period for personal data, we consider the amount, nature, and sensitivity of the personal data, the potential risk of harm from unauthorised use or disclosure of your personal data, the purposes for which we process your personal data and whether we can achieve those purposes through other means, and the applicable legal requirements. 
In some circumstances we may anonymise your personal information so that it can no longer be associated with you, in which case we may use such information without further notice to you. Once you are no longer a member of the organisation we will retain and securely destroy your personal information in accordance with applicable laws and regulations.


\section{Rights of access, correction, erasure, and restriction}

Please find below information about your rights regarding the access, correction, erasure and restriction of data which we hold about you. If you wish to enquire further about exercising these rights, please contact us at our designated contact address above.

\subsection*{Your duty to inform us of changes}
It is important that the personal information we hold about you is accurate and current. Please keep us informed if your personal information changes during your relationship with us. 

\subsection*{Your rights in connection with personal information}

\subsubsection*{Request access to your personal information}
Under certain circumstances, by law you have the right to:
Request access to your personal information (commonly known as a "data subject access request"). This enables you to receive a copy of the personal information we hold about you and to check that we are lawfully processing it.

\subsubsection*{Request correction of the personal information that we hold about you} This enables you to have any incomplete or inaccurate information we hold about you corrected.

\subsubsection*{Request erasure of your personal information}

This enables you to ask us to delete or remove personal information where there is no good reason for us continuing to process it. You also have the right to ask us to delete or remove your personal information where you have exercised your right to object to processing (see below).

\subsubsection*{Object to us processing your information}
Object to processing of your personal information where we are relying on a legitimate interest (or those of a third party) and there is something about your particular situation which makes you want to object to processing on this ground. You also have the right to object where we are processing your personal information for direct marketing purposes.

\subsubsection*{Request the restriction of processing of your personal information}
This enables you to ask us to suspend the processing of personal information about you, for example if you want us to establish its accuracy or the reason for processing it. 
If you want to review, verify, correct or request erasure of your personal information, or object to the processing of your personal data, or request that we transfer a copy of your personal information to another party, please contact us using the above contact details.

\subsubsection*{No fee usually required}

You will not have to pay a fee to access your personal information (or to exercise any of the other rights). However, we may charge a reasonable fee if your request for access is clearly unfounded or excessive. Alternatively, we may refuse to comply with the request in such circumstances.
What we may need from you
We may need to request specific information from you to help us confirm your identity and ensure your right to access the information (or to exercise any of your other rights). This is another appropriate security measure to ensure that personal information is not disclosed to any person who has no right to receive it.

\subsubsection*{Right to withdraw consent}

In the limited circumstances where you may have provided your consent to the collection, processing and transfer of your personal information for a specific purpose, you have the right to withdraw your consent for that specific processing at any time. To withdraw your consent, please contact us using the above contact details.  Once we have received notification that you have withdrawn your consent, we will no longer process your information for the purpose or purposes you originally agreed to, unless we have another legitimate basis for doing so in law.

\section*{Changes to this privacy notice}
We reserve the right to update this privacy notice at any time, and we will provide you with a new privacy notice when we make any substantial updates. We may also notify you in other ways from time to time about the processing of your personal information. 

If you have any questions about this privacy notice, please contact us using the above contact details.



\end{document}
  
